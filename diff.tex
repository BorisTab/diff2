\documentclass[12pt,a4paper]{scrartcl}
\usepackage[utf8]{inputenc}
\usepackage{resizegather}
\usepackage[english,russian]{babel}
\usepackage{indentfirst}
\usepackage{misccorr}
\usepackage{graphicx}
\usepackage{amsmath}
\begin{document}
Легким движением руки дифференцируем данную функцию 1 раз(а):
\begin{gather}\label{eq:f2a56b10}f  =  \cos{ x } \end{gather}
Абсолютно не думая можно получить следующее:
\begin{gather}\label{eq:f2a56b10}g  =  \cos{ x } \end{gather}
Нетрудно заметить, что:
\begin{gather}\label{eq:f2a56ae0}g  = x\end{gather}
Очевидным переходом получаем:
\begin{gather}\label{eq:f2a56c90}g^\prime  = 1\end{gather}
Далее просто получем:
\begin{gather}\label{eq:f2a56c60}g^\prime  = 1 \cdot  \left(- \sin{ x } \right) \end{gather}
Абсолютно не думая можно получить следующее:
\begin{gather}\label{eq:f2a56c90}g  =  \cos{ x } \end{gather}
Далее просто получем:
\begin{gather}\label{eq:f2a56cc0}g  = x\end{gather}
Нетрудно заметить, что:
\begin{gather}\label{eq:f2a57260}g^\prime  = 1\end{gather}
Очевидным переходом получаем:
\begin{gather}\label{eq:f2a57230}g^\prime  = 1 \cdot  \left(- \sin{ x } \right) \end{gather}
Тут придется приложить усилее, но решить можно:
\begin{gather}\label{eq:f2a57260}g  =  \cos{ x } \end{gather}
Нетрудно заметить, что:
\begin{gather}\label{eq:f2a57290}g  = x\end{gather}
Очевидным переходом получаем:
\begin{gather}\label{eq:f2a57440}g^\prime  = 1\end{gather}
Очевидным переходом получаем:
\begin{gather}\label{eq:f2a57410}g^\prime  = 1 \cdot  \left(- \sin{ x } \right) \end{gather}
Далее просто получем:
\begin{gather}\label{eq:f2a57290}g  =  \left(- \sin{ x } \right) \end{gather}
Тут придется приложить усилее, но решить можно:
\begin{gather}\label{eq:f2a57440}g  = \left(-1\right)\end{gather}
Абсолютно не думая можно получить следующее:
\begin{gather}\label{eq:f2a57500}g^\prime  = 0\end{gather}
Нетрудно заметить, что:
\begin{gather}\label{eq:f2a57470}g  =  \sin{ x } \end{gather}
Далее просто получем:
\begin{gather}\label{eq:f2a574a0}g  = x\end{gather}
Нетрудно заметить, что:
\begin{gather}\label{eq:f2a57680}g^\prime  = 1\end{gather}
Очевидным переходом получаем:
\begin{gather}\label{eq:f2a57650}g^\prime  = 1 \cdot  \cos{ x } \end{gather}
Нетрудно заметить, что:
\begin{gather}\label{eq:f2a57260}g^\prime  = 0 \cdot  \sin{ x }  +  \left(-1 \cdot  \cos{ x } \right) \end{gather}
Нетрудно заметить, что:
\begin{gather}\label{eq:f2a57680}g  =  \cos{ x } \end{gather}
Абсолютно не думая можно получить следующее:
\begin{gather}\label{eq:f2a574d0}g  = x\end{gather}
Очевидным переходом получаем:
\begin{gather}\label{eq:f2a577d0}g^\prime  = 1\end{gather}
Тут придется приложить усилее, но решить можно:
\begin{gather}\label{eq:f2a577a0}g^\prime  = 1 \cdot  \left(- \sin{ x } \right) \end{gather}
Очевидным переходом получаем:
\begin{gather}\label{eq:f2a574d0}g  =  \left(- \sin{ x } \right) \end{gather}
Далее просто получем:
\begin{gather}\label{eq:f2a577d0}g  = \left(-1\right)\end{gather}
Очевидным переходом получаем:
\begin{gather}\label{eq:f2a57890}g^\prime  = 0\end{gather}
Абсолютно не думая можно получить следующее:
\begin{gather}\label{eq:f2a57800}g  =  \sin{ x } \end{gather}
Очевидным переходом получаем:
\begin{gather}\label{eq:f2a57830}g  = x\end{gather}
Далее просто получем:
\begin{gather}\label{eq:f2a57a10}g^\prime  = 1\end{gather}
Тут придется приложить усилее, но решить можно:
\begin{gather}\label{eq:f2a579e0}g^\prime  = 1 \cdot  \cos{ x } \end{gather}
Очевидным переходом получаем:
\begin{gather}\label{eq:f2a57680}g^\prime  = 0 \cdot  \sin{ x }  +  \left(-1 \cdot  \cos{ x } \right) \end{gather}
Очевидным переходом получаем:
\begin{gather}\label{eq:f2a57830}g  =  \left(- \cos{ x } \right) \end{gather}
Абсолютно не думая можно получить следующее:
\begin{gather}\label{eq:f2a577d0}g  = \left(-1\right)\end{gather}
Тут придется приложить усилее, но решить можно:
\begin{gather}\label{eq:f2a578f0}g^\prime  = 0\end{gather}
Очевидным переходом получаем:
\begin{gather}\label{eq:f2a57a10}g  =  \cos{ x } \end{gather}
Абсолютно не думая можно получить следующее:
\begin{gather}\label{eq:f2a57860}g  = x\end{gather}
Нетрудно заметить, что:
\begin{gather}\label{eq:f2a57b90}g^\prime  = 1\end{gather}
Нетрудно заметить, что:
\begin{gather}\label{eq:f2a57b60}g^\prime  = 1 \cdot  \left(- \sin{ x } \right) \end{gather}
Очевидным переходом получаем:
\begin{gather}\label{eq:f2a574d0}g^\prime  = 0 \cdot  \cos{ x }  +  \left(-1 \cdot  \left(- \sin{ x } \right) \right) \end{gather}
Тут придется приложить усилее, но решить можно:
\begin{gather}\label{eq:f2a57b90}g  =  \cos{ x } \end{gather}
Абсолютно не думая можно получить следующее:
\begin{gather}\label{eq:f2a578c0}g  = x\end{gather}
Нетрудно заметить, что:
\begin{gather}\label{eq:f2a57cb0}g^\prime  = 1\end{gather}
Тут придется приложить усилее, но решить можно:
\begin{gather}\label{eq:f2a57c80}g^\prime  = 1 \cdot  \left(- \sin{ x } \right) \end{gather}
Тут придется приложить усилее, но решить можно:
\begin{gather}\label{eq:f2a578c0}g  =  \left(- \sin{ x } \right) \end{gather}
Очевидным переходом получаем:
\begin{gather}\label{eq:f2a57cb0}g  = \left(-1\right)\end{gather}
Абсолютно не думая можно получить следующее:
\begin{gather}\label{eq:f2a57d70}g^\prime  = 0\end{gather}
Тут придется приложить усилее, но решить можно:
\begin{gather}\label{eq:f2a57ce0}g  =  \sin{ x } \end{gather}
Далее просто получем:
\begin{gather}\label{eq:f2a57d10}g  = x\end{gather}
Далее просто получем:
\begin{gather}\label{eq:f2a57ef0}g^\prime  = 1\end{gather}
Абсолютно не думая можно получить следующее:
\begin{gather}\label{eq:f2a57ec0}g^\prime  = 1 \cdot  \cos{ x } \end{gather}
Нетрудно заметить, что:
\begin{gather}\label{eq:f2a57b90}g^\prime  = 0 \cdot  \sin{ x }  +  \left(-1 \cdot  \cos{ x } \right) \end{gather}
Нетрудно заметить, что:
\begin{gather}\label{eq:f2a57d10}g  =  \left(- \cos{ x } \right) \end{gather}
Далее просто получем:
\begin{gather}\label{eq:f2a57cb0}g  = \left(-1\right)\end{gather}
Нетрудно заметить, что:
\begin{gather}\label{eq:f2a57dd0}g^\prime  = 0\end{gather}
Абсолютно не думая можно получить следующее:
\begin{gather}\label{eq:f2a57ef0}g  =  \cos{ x } \end{gather}
Очевидным переходом получаем:
\begin{gather}\label{eq:f2a57d40}g  = x\end{gather}
Далее просто получем:
\begin{gather}\label{eq:f2a58070}g^\prime  = 1\end{gather}
Нетрудно заметить, что:
\begin{gather}\label{eq:f2a58040}g^\prime  = 1 \cdot  \left(- \sin{ x } \right) \end{gather}
Нетрудно заметить, что:
\begin{gather}\label{eq:f2a578c0}g^\prime  = 0 \cdot  \cos{ x }  +  \left(-1 \cdot  \left(- \sin{ x } \right) \right) \end{gather}
Далее просто получем:
\begin{gather}\label{eq:f2a57d40}g  =  \left(- \left(- \sin{ x } \right) \right) \end{gather}
Далее просто получем:
\begin{gather}\label{eq:f2a57cb0}g  = \left(-1\right)\end{gather}
Тут придется приложить усилее, но решить можно:
\begin{gather}\label{eq:f2a57ef0}g^\prime  = 0\end{gather}
Очевидным переходом получаем:
\begin{gather}\label{eq:f2a58070}g  =  \left(- \sin{ x } \right) \end{gather}
Нетрудно заметить, что:
\begin{gather}\label{eq:f2a57da0}g  = \left(-1\right)\end{gather}
Далее просто получем:
\begin{gather}\label{eq:f2a58220}g^\prime  = 0\end{gather}
Нетрудно заметить, что:
\begin{gather}\label{eq:f2a57d70}g  =  \sin{ x } \end{gather}
Тут придется приложить усилее, но решить можно:
\begin{gather}\label{eq:f2a57ce0}g  = x\end{gather}
Абсолютно не думая можно получить следующее:
\begin{gather}\label{eq:f2a583a0}g^\prime  = 1\end{gather}
Очевидным переходом получаем:
\begin{gather}\label{eq:f2a58370}g^\prime  = 1 \cdot  \cos{ x } \end{gather}
Тут придется приложить усилее, но решить можно:
\begin{gather}\label{eq:f2a581c0}g^\prime  = 0 \cdot  \sin{ x }  +  \left(-1 \cdot  \cos{ x } \right) \end{gather}
Далее просто получем:
\begin{gather}\label{eq:f2a57d10}g^\prime  = 0 \cdot  \left(- \sin{ x } \right)  +  \left(-0 \cdot  \sin{ x }  +  \left(-1 \cdot  \cos{ x } \right) \right) \end{gather}
Очевидным переходом получаем:
\begin{gather}\label{eq:f2a58280}g  =  \cos{ x } \end{gather}
Далее просто получем:
\begin{gather}\label{eq:f2a58220}g  = x\end{gather}
Тут придется приложить усилее, но решить можно:
\begin{gather}\label{eq:f2a583d0}g^\prime  = 1\end{gather}
Очевидным переходом получаем:
\begin{gather}\label{eq:f2a581f0}g^\prime  = 1 \cdot  \left(- \sin{ x } \right) \end{gather}
Тут придется приложить усилее, но решить можно:
\begin{gather}\label{eq:f2a58220}g  =  \left(- \sin{ x } \right) \end{gather}
Тут придется приложить усилее, но решить можно:
\begin{gather}\label{eq:f2a583d0}g  = \left(-1\right)\end{gather}
Абсолютно не думая можно получить следующее:
\begin{gather}\label{eq:f2a58490}g^\prime  = 0\end{gather}
Далее просто получем:
\begin{gather}\label{eq:f2a58400}g  =  \sin{ x } \end{gather}
Очевидным переходом получаем:
\begin{gather}\label{eq:f2a58430}g  = x\end{gather}
Абсолютно не думая можно получить следующее:
\begin{gather}\label{eq:f2a58610}g^\prime  = 1\end{gather}
Нетрудно заметить, что:
\begin{gather}\label{eq:f2a585e0}g^\prime  = 1 \cdot  \cos{ x } \end{gather}
Абсолютно не думая можно получить следующее:
\begin{gather}\label{eq:f2a58280}g^\prime  = 0 \cdot  \sin{ x }  +  \left(-1 \cdot  \cos{ x } \right) \end{gather}
Абсолютно не думая можно получить следующее:
\begin{gather}\label{eq:f2a58430}g  =  \left(- \cos{ x } \right) \end{gather}
Тут придется приложить усилее, но решить можно:
\begin{gather}\label{eq:f2a583d0}g  = \left(-1\right)\end{gather}
Абсолютно не думая можно получить следующее:
\begin{gather}\label{eq:f2a584f0}g^\prime  = 0\end{gather}
Нетрудно заметить, что:
\begin{gather}\label{eq:f2a58610}g  =  \cos{ x } \end{gather}
Тут придется приложить усилее, но решить можно:
\begin{gather}\label{eq:f2a58460}g  = x\end{gather}
Тут придется приложить усилее, но решить можно:
\begin{gather}\label{eq:f2a58790}g^\prime  = 1\end{gather}
Очевидным переходом получаем:
\begin{gather}\label{eq:f2a58760}g^\prime  = 1 \cdot  \left(- \sin{ x } \right) \end{gather}
Далее просто получем:
\begin{gather}\label{eq:f2a58220}g^\prime  = 0 \cdot  \cos{ x }  +  \left(-1 \cdot  \left(- \sin{ x } \right) \right) \end{gather}
Нетрудно заметить, что:
\begin{gather}\label{eq:f2a58460}g  =  \left(- \left(- \sin{ x } \right) \right) \end{gather}
Абсолютно не думая можно получить следующее:
\begin{gather}\label{eq:f2a583d0}g  = \left(-1\right)\end{gather}
Тут придется приложить усилее, но решить можно:
\begin{gather}\label{eq:f2a58610}g^\prime  = 0\end{gather}
Очевидным переходом получаем:
\begin{gather}\label{eq:f2a58790}g  =  \left(- \sin{ x } \right) \end{gather}
Нетрудно заметить, что:
\begin{gather}\label{eq:f2a584c0}g  = \left(-1\right)\end{gather}
Нетрудно заметить, что:
\begin{gather}\label{eq:f2a58940}g^\prime  = 0\end{gather}
Тут придется приложить усилее, но решить можно:
\begin{gather}\label{eq:f2a58490}g  =  \sin{ x } \end{gather}
Тут придется приложить усилее, но решить можно:
\begin{gather}\label{eq:f2a58400}g  = x\end{gather}
Далее просто получем:
\begin{gather}\label{eq:f2a58ac0}g^\prime  = 1\end{gather}
Далее просто получем:
\begin{gather}\label{eq:f2a58a90}g^\prime  = 1 \cdot  \cos{ x } \end{gather}
Тут придется приложить усилее, но решить можно:
\begin{gather}\label{eq:f2a588e0}g^\prime  = 0 \cdot  \sin{ x }  +  \left(-1 \cdot  \cos{ x } \right) \end{gather}
Абсолютно не думая можно получить следующее:
\begin{gather}\label{eq:f2a58430}g^\prime  = 0 \cdot  \left(- \sin{ x } \right)  +  \left(-0 \cdot  \sin{ x }  +  \left(-1 \cdot  \cos{ x } \right) \right) \end{gather}
Очевидным переходом получаем:
\begin{gather}\label{eq:f2a589a0}g  =  \left(- \left(- \cos{ x } \right) \right) \end{gather}
Нетрудно заметить, что:
\begin{gather}\label{eq:f2a58940}g  = \left(-1\right)\end{gather}
Очевидным переходом получаем:
\begin{gather}\label{eq:f2a58790}g^\prime  = 0\end{gather}
Очевидным переходом получаем:
\begin{gather}\label{eq:f2a587c0}g  =  \left(- \cos{ x } \right) \end{gather}
Очевидным переходом получаем:
\begin{gather}\label{eq:f2a58820}g  = \left(-1\right)\end{gather}
Очевидным переходом получаем:
\begin{gather}\label{eq:f2a58af0}g^\prime  = 0\end{gather}
Очевидным переходом получаем:
\begin{gather}\label{eq:f2a58850}g  =  \cos{ x } \end{gather}
Нетрудно заметить, что:
\begin{gather}\label{eq:f2a587f0}g  = x\end{gather}
Нетрудно заметить, что:
\begin{gather}\label{eq:f2a58cd0}g^\prime  = 1\end{gather}
Далее просто получем:
\begin{gather}\label{eq:f2a58ca0}g^\prime  = 1 \cdot  \left(- \sin{ x } \right) \end{gather}
Нетрудно заметить, что:
\begin{gather}\label{eq:f2a58490}g^\prime  = 0 \cdot  \cos{ x }  +  \left(-1 \cdot  \left(- \sin{ x } \right) \right) \end{gather}
Нетрудно заметить, что:
\begin{gather}\label{eq:f2a58460}g^\prime  = 0 \cdot  \left(- \cos{ x } \right)  +  \left(-0 \cdot  \cos{ x }  +  \left(-1 \cdot  \left(- \sin{ x } \right) \right) \right) \end{gather}
Разложение функции в тейлора с точность o($x^5$) в точке 0:
\begin{gather}\label{eq:3}f(x) = 1 + \frac{0}{1!} \cdot \left(x-0 \right)^1 + \frac{-1}{2!} \cdot \left(x-0 \right)^2 + \frac{-0}{3!} \cdot \left(x-0 \right)^3 + \frac{1}{4!} \cdot \left(x-0 \right)^4 + \frac{0}{5!} \cdot \left(x-0 \right)^5 + o((x-0)^5)\end{gather}
Из вышесказанного очевидным образом получаем ответ:
\begin{gather}\label{eq:f2a56ba0}f^\prime  =  \left(- \sin{ x } \right) \end{gather}
\end{document}