\documentclass[12pt,a4paper]{scrartcl}
\usepackage[utf8]{inputenc}
\usepackage{resizegather}
\usepackage[english,russian]{babel}
\usepackage{indentfirst}
\usepackage{misccorr}
\usepackage{graphicx}
\usepackage{amsmath}
\begin{document}
Легким движением руки дифференцируем данную функцию 1 раз(а):
\begin{gather}\label{eq:1}f  = 3 \cdot x ^ {2} +  \cos{ \left( \sin{ \left(\frac{2 \cdot x + 3}{x ^ {2}}\right) } \right) } \end{gather}
Абсолютно не думая можно получить следующее:
\begin{gather}\label{eq:1}g  = 3 \cdot x ^ {2} +  \cos{ \left( \sin{ \left(\frac{2 \cdot x + 3}{x ^ {2}}\right) } \right) } \end{gather}
Нетрудно заметить, что:
\begin{gather}\label{eq:1}g  = 3 \cdot x ^ {2}\end{gather}
Очевидным переходом получаем:
\begin{gather}\label{eq:1}g  = 3\end{gather}
Далее просто получем:
\begin{gather}\label{eq:1}g^\prime  = 0\end{gather}
Абсолютно не думая можно получить следующее:
\begin{gather}\label{eq:1}g  = x ^ {2}\end{gather}
Далее просто получем:
\begin{gather}\label{eq:1}g  = x\end{gather}
Нетрудно заметить, что:
\begin{gather}\label{eq:1}g^\prime  = 1\end{gather}
Очевидным переходом получаем:
\begin{gather}\label{eq:1}g^\prime  = 1 \cdot 2 \cdot x ^ {2 - 1}\end{gather}
Тут придется приложить усилее, но решить можно:
\begin{gather}\label{eq:1}g^\prime  = 0 \cdot x ^ {2} + 3 \cdot 1 \cdot 2 \cdot x ^ {2 - 1}\end{gather}
Нетрудно заметить, что:
\begin{gather}\label{eq:1}g  =  \cos{ \left( \sin{ \left(\frac{2 \cdot x + 3}{x ^ {2}}\right) } \right) } \end{gather}
Очевидным переходом получаем:
\begin{gather}\label{eq:1}g  =  \sin{ \left(\frac{2 \cdot x + 3}{x ^ {2}}\right) } \end{gather}
Очевидным переходом получаем:
\begin{gather}\label{eq:1}g  = \frac{2 \cdot x + 3}{x ^ {2}}\end{gather}
Далее просто получем:
\begin{gather}\label{eq:1}g  = 2 \cdot x + 3\end{gather}
Тут придется приложить усилее, но решить можно:
\begin{gather}\label{eq:1}g  = 2 \cdot x\end{gather}
Абсолютно не думая можно получить следующее:
\begin{gather}\label{eq:1}g  = 2\end{gather}
Нетрудно заметить, что:
\begin{gather}\label{eq:1}g^\prime  = 0\end{gather}
Далее просто получем:
\begin{gather}\label{eq:1}g  = x\end{gather}
Нетрудно заметить, что:
\begin{gather}\label{eq:1}g^\prime  = 1\end{gather}
Очевидным переходом получаем:
\begin{gather}\label{eq:1}g^\prime  = 0 \cdot x + 2 \cdot 1\end{gather}
Нетрудно заметить, что:
\begin{gather}\label{eq:1}g  = 3\end{gather}
Нетрудно заметить, что:
\begin{gather}\label{eq:1}g^\prime  = 0\end{gather}
Абсолютно не думая можно получить следующее:
\begin{gather}\label{eq:1}g^\prime  = 0 \cdot x + 2 \cdot 1 + 0\end{gather}
Очевидным переходом получаем:
\begin{gather}\label{eq:1}g  = x ^ {2}\end{gather}
Тут придется приложить усилее, но решить можно:
\begin{gather}\label{eq:1}g  = x\end{gather}
Очевидным переходом получаем:
\begin{gather}\label{eq:1}g^\prime  = 1\end{gather}
Далее просто получем:
\begin{gather}\label{eq:1}g^\prime  = 1 \cdot 2 \cdot x ^ {2 - 1}\end{gather}
Очевидным переходом получаем:
\begin{gather}\label{eq:1}g^\prime  = \frac{\left(0 \cdot x + 2 \cdot 1 + 0\right) \cdot x ^ {2} - \left(2 \cdot x + 3\right) \cdot 1 \cdot 2 \cdot x ^ {2 - 1}}{{ \left(x ^ {2}\right) } ^ {2}}\end{gather}
Абсолютно не думая можно получить следующее:
\begin{gather}\label{eq:1}g^\prime  = \frac{\left(0 \cdot x + 2 \cdot 1 + 0\right) \cdot x ^ {2} - \left(2 \cdot x + 3\right) \cdot 1 \cdot 2 \cdot x ^ {2 - 1}}{{ \left(x ^ {2}\right) } ^ {2}} \cdot  \cos{ \left(\frac{2 \cdot x + 3}{x ^ {2}}\right) } \end{gather}
Очевидным переходом получаем:
\begin{gather}\label{eq:1}g^\prime  = \frac{\left(0 \cdot x + 2 \cdot 1 + 0\right) \cdot x ^ {2} - \left(2 \cdot x + 3\right) \cdot 1 \cdot 2 \cdot x ^ {2 - 1}}{{ \left(x ^ {2}\right) } ^ {2}} \cdot  \cos{ \left(\frac{2 \cdot x + 3}{x ^ {2}}\right) }  \cdot  \left(- \sin{ \left( \sin{ \left(\frac{2 \cdot x + 3}{x ^ {2}}\right) } \right) } \right) \end{gather}
Далее просто получем:
\begin{gather}\label{eq:1}g^\prime  = 0 \cdot x ^ {2} + 3 \cdot 1 \cdot 2 \cdot x ^ {2 - 1} + \frac{\left(0 \cdot x + 2 \cdot 1 + 0\right) \cdot x ^ {2} - \left(2 \cdot x + 3\right) \cdot 1 \cdot 2 \cdot x ^ {2 - 1}}{{ \left(x ^ {2}\right) } ^ {2}} \cdot  \cos{ \left(\frac{2 \cdot x + 3}{x ^ {2}}\right) }  \cdot  \left(- \sin{ \left( \sin{ \left(\frac{2 \cdot x + 3}{x ^ {2}}\right) } \right) } \right) \end{gather}
Из вышесказанного очевидным образом получаем ответ:
\begin{gather}\label{eq:1}f^\prime  = 3 \cdot 2 \cdot x + \frac{2 \cdot x ^ {2} - \left(2 \cdot x + 3\right) \cdot 2 \cdot x}{{ \left(x ^ {2}\right) } ^ {2}} \cdot  \cos{ \left(\frac{2 \cdot x + 3}{x ^ {2}}\right) }  \cdot  \left(- \sin{ \left( \sin{ \left(\frac{2 \cdot x + 3}{x ^ {2}}\right) } \right) } \right) \end{gather}
\end{document}