\documentclass[12pt,a4paper]{scrartcl}
\usepackage[utf8]{inputenc}
\usepackage{resizegather}
\usepackage[english,russian]{babel}
\usepackage{indentfirst}
\usepackage{misccorr}
\usepackage{graphicx}
\usepackage{amsmath}
\begin{document}
Легким движением руки дифференцируем данную функцию 1 раз(а):
\begin{gather}\label{eq:55574b10}f  =  \cos{ x } \end{gather}
Абсолютно не думая можно получить следующее:
\begin{gather}\label{eq:55574b10}g  =  \cos{ x } \end{gather}
Нетрудно заметить, что:
\begin{gather}\label{eq:55574ae0}g  = x\end{gather}
Очевидным переходом получаем:
\begin{gather}\label{eq:55574c30}g^\prime  = 1\end{gather}
Далее просто получем:
\begin{gather}\label{eq:55574c00}g^\prime  = 1 \cdot  \left(- \sin{ x } \right) \end{gather}
Абсолютно не думая можно получить следующее:
\begin{gather}\label{eq:55574b10}g  =  \cos{ x } \end{gather}
Далее просто получем:
\begin{gather}\label{eq:55574ae0}g  = x\end{gather}
Нетрудно заметить, что:
\begin{gather}\label{eq:55575140}g^\prime  = 1\end{gather}
Очевидным переходом получаем:
\begin{gather}\label{eq:55575110}g^\prime  = 1 \cdot  \left(- \sin{ x } \right) \end{gather}
Тут придется приложить усилее, но решить можно:
\begin{gather}\label{eq:55574b10}g  =  \cos{ x } \end{gather}
Нетрудно заметить, что:
\begin{gather}\label{eq:55574ae0}g  = x\end{gather}
Очевидным переходом получаем:
\begin{gather}\label{eq:55575260}g^\prime  = 1\end{gather}
Очевидным переходом получаем:
\begin{gather}\label{eq:55575230}g^\prime  = 1 \cdot  \left(- \sin{ x } \right) \end{gather}
Далее просто получем:
\begin{gather}\label{eq:55574ae0}g  =  \left(- \sin{ x } \right) \end{gather}
Тут придется приложить усилее, но решить можно:
\begin{gather}\label{eq:55575260}g  = \left(-1\right)\end{gather}
Абсолютно не думая можно получить следующее:
\begin{gather}\label{eq:55575320}g^\prime  = 0\end{gather}
Нетрудно заметить, что:
\begin{gather}\label{eq:55575290}g  =  \sin{ x } \end{gather}
Далее просто получем:
\begin{gather}\label{eq:555752c0}g  = x\end{gather}
Нетрудно заметить, что:
\begin{gather}\label{eq:555754a0}g^\prime  = 1\end{gather}
Очевидным переходом получаем:
\begin{gather}\label{eq:55575470}g^\prime  = 1 \cdot  \cos{ x } \end{gather}
Нетрудно заметить, что:
\begin{gather}\label{eq:55574b10}g^\prime  = 0 \cdot  \sin{ x }  +  \left(-1 \cdot  \cos{ x } \right) \end{gather}
Нетрудно заметить, что:
\begin{gather}\label{eq:55574ae0}g  =  \left(- \sin{ x } \right) \end{gather}
Абсолютно не думая можно получить следующее:
\begin{gather}\label{eq:55575260}g  = \left(-1\right)\end{gather}
Очевидным переходом получаем:
\begin{gather}\label{eq:55575320}g^\prime  = 0\end{gather}
Тут придется приложить усилее, но решить можно:
\begin{gather}\label{eq:55575290}g  =  \sin{ x } \end{gather}
Очевидным переходом получаем:
\begin{gather}\label{eq:555752c0}g  = x\end{gather}
Далее просто получем:
\begin{gather}\label{eq:55575620}g^\prime  = 1\end{gather}
Очевидным переходом получаем:
\begin{gather}\label{eq:555755f0}g^\prime  = 1 \cdot  \cos{ x } \end{gather}
Абсолютно не думая можно получить следующее:
\begin{gather}\label{eq:55575350}g^\prime  = 0 \cdot  \sin{ x }  +  \left(-1 \cdot  \cos{ x } \right) \end{gather}
Очевидным переходом получаем:
\begin{gather}\label{eq:555752c0}g  =  \left(- \cos{ x } \right) \end{g