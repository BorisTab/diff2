\documentclass[12pt,a4paper]{scrartcl}
\usepackage[utf8]{inputenc}
\usepackage{resizegather}
\usepackage[english,russian]{babel}
\usepackage{indentfirst}
\usepackage{misccorr}
\usepackage{graphicx}
\usepackage{amsmath}
\begin{document}
Легким движением руки дифференцируем данную функцию 1 раз(а):
\begin{gather}\label{eq:1}f  =  \cos{ x } \end{gather}
Абсолютно не думая можно получить следующее:
\begin{gather}\label{eq:1}g  =  \cos{ x } \end{gather}
Нетрудно заметить, что:
\begin{gather}\label{eq:1}g  = x\end{gather}
Очевидным переходом получаем:
\begin{gather}\label{eq:1}g^\prime  = 1\end{gather}
Далее просто получем:
\begin{gather}\label{eq:1}g^\prime  = 1 \cdot  \left(- \sin{ x } \right) \end{gather}
Из вышесказанного очевидным образом получаем ответ:
\begin{gather}\label{eq:1}f^\prime  =  \left(- \sin{ x } \right) \end{gather}
\end{document}